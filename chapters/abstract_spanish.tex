\newpage
\prefaceChapterTitle{resumen}{20}{25}{35}

\normalsize
\libertineNormal

El objetivo de esta tesis es desarrollar una metodología capaz de controlar múltiples fuentes de datos, incluyendo imágenes y nubes de puntos 3D, así como capaz de corregir y fusionar dichas fuentes para aplicarlas a la monitorización, predicción y optimización de procesos del mundo real. No obstante, trabajar con datos adquiridos mediante sensores es tedioso en muchos aspectos, entre los que se incluyen las fases de adquisición, marcado de puntos de control o la clasificación de puntos. Para evitar estas tareas, se propone generar datos sintéticos a partir de escenarios modelados de manera realista, evitando así adquirir tecnología con un elevado coste y construyendo conjuntos de datos de gran tamaño de manera muy eficiente. Además, los modelos de estos escenarios se pueden relacionar con etiquetas semánticas y materiales, entre otras propiedades. A diferencia del etiquetado manual, los conjuntos de datos sintéticos no incluyen información errónea que puede inducir a error a los algoritmos que utilicen dichos datos.

Las imágenes procedentes de teledetección, a pesar de presentar diferencias de intensidad notables, se pueden fusionar maximizando la correlación entre ellas. Esta tesis emplea el algoritmo de registro de imágenes conocido como \textit{Enhanced Correlation Coefficient} para el registro de imágenes multiespectrales, térmicas y en el espectro visible. Después, la proyección de información multiespectral y térmica al 3D se hace transfiriendo estos datos a una nube de puntos RGB muy densa reconstruida con fotogrametría. El objetivo que se persigue utilizando una proyección en lugar de una reconstrucción directa es obtener nubes de puntos densas y con una geometría precisa a partir de imágenes de muy baja resolución. Además, esta metodología es notablemente más eficiente que la que integran algunas soluciones comerciales. La calidad de los datos radiométricos se garantiza teniendo en cuenta la oclusión, y minimizando la disilimitud entre la agregación de un conjunto de datos y dichos datos.

Por otra parte , los datos hiperespectrales se proyectaron a nubes de puntos voxelizadas (2.5D) mediante una metodología adaptada al escaneo de tipo \textit{push broom}. En primer lugar, se corrige la geometría de las imágenes hiperespectrales y se solapan para componer un ortomosaico. Después, éste se proyecta en una nube voxelizada. Debido al gran volumen del hipercubo resultante, se comprimen los datos radiométricos mediante una representación basada en \textit{stacks}. Finalmente, se renderiza en tiempo real el hipercubo comprimido mediante la construcción de la imagen en múltiples fotogramas, de manera iterativa.
 
En comparación con los datos previos, los datos sintéticos se centran en la tecnología LiDAR. La base de esta simulación es la indexación de escenarios con un alto nivel de detalle mediante estructuras de datos diseñadas para resolver rápidamente intersecciones entre rayos y triángulos. Sobre esta base, se incluyen errores sistemáticos y aleatorios, tales como puntos anómalos, ruido en la trayectoria de un pulso o pérdidas de retornos, entre otros. Además, se generan escenarios procedurales enriquecidos con datos semánticos y materiales, que posibilitan la construcción de grandes conjuntos de datos LiDAR. Los escaneos aéreos y terrestres se parametrizan para poder incluir un amplio número de sensores comerciales. Además, integran múltiples patrones de escaneo y simulan la intensidad utilizando muestras obtenidas con un goniofotómetro, publicadas en una base de datos de BRDFs. Por último, también se contemplan longitudes de onda diferentes en el sensor LiDAR, incluyendo el escaneo batimétrico, y se simulan múltiples retornos.

Esta tesis concluye mostrando los beneficios de los datos fusionados y generados de manera sintética mediante tres casos de estudio diferentes. La simulación LiDAR se emplea para optimizar el escaneado de edificios mediante búsquedas locales, que mejora las posiciones de escaneo, y algoritmos genéticos, que minimizan el número de escaneos necesarios. Estas metaheurísticas son guiadas por cuatro funciones objetivo que evalúan precisión, cobertura, nivel de detalle y solapamiento de los escaneos propuestos. Después, nubes de puntos térmicas y mapas ortorectificados, también térmicos, se emplean en la identificación de posibles restos enterrados y la reconstrucción de la estructura de un yacimiento arqueológico. Se muestra así el potencial de la teledetección para preservar el patrimonio histórico. Finalmente, los datos hiperespectrales se corrigen y transforman para entrenar una red neuronal convolucional con el objetivo de clasificar variedades de vid.
